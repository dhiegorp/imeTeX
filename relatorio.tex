\documentclass[
  % -- opções da classe memoir --
  12pt,                 % tamanho da fonte
  %openright,           % capítulos começam em pág ímpar (insere página vazia caso preciso)
  %twoside,             % para impressão em verso e anverso. Oposto a oneside
  oneside,              % para impressão em verso e anverso. Oposto a oneside
  a4paper,              % tamanho do papel.
  % -- opções da classe abntex2 --
  %chapter=TITLE,       % títulos de capítulos convertidos em letras maiúsculas
  %section=TITLE,       % títulos de seções convertidos em letras maiúsculas
  %subsection=TITLE,    % títulos de subseções convertidos em letras maiúsculas
  %subsubsection=TITLE, % títulos de subsubseções convertidos em letras maiúsculas
  % -- opções do pacote babel --
  english,              % idioma adicional para hifenização
  brazil,               % o último idioma é o principal do documento
  % -- opções da classe ime-abntex2 --
  %brasao,              % descomente para habilitar o brasão no topo da capa
]{ime-abntex2}


% ---
% PACOTES
% ---

% ---
% Pacotes fundamentais para o relatório em questão
% ---
\usepackage[utf8]{inputenc} % Codificacao do documento (conversão automática dos acentos)
\usepackage{cmap}           % Mapear caracteres especiais no PDF
\usepackage{lmodern}        % Usa a fonte Latin Modern
%\usepackage{amsmath}       % Para usar fontes ams
%\usepackage{amsfonts}      % Para usar fontes ams
%\usepackage{times}         % Usa a fonte Times
\usepackage[T1]{fontenc}    % Selecao de codigos de fonte.
%\usepackage{lastpage}      % Usado pela Ficha catalográfica
\usepackage{indentfirst}    % Indenta o primeiro parágrafo de cada seção.
\usepackage{color}          % Controle das cores
\usepackage{graphicx}       % Inclusão de gráficos
% ---

% ---
% Pacotes adicionais, usados apenas no âmbito do documento em questão
% ---
\usepackage{lipsum} % para geração de dummy text
% ---

% ---
% Pacotes de citações
% ---
\usepackage[alf]{abntex2cite} % Citações padrão ABNT


% ---
% Informações de dados para CAPA e FOLHA DE ROSTO
% ---
\titulo{O Título do Seu Trabalho Aqui}
\autor{
  Fulano da Silva\\
  Ciclano de Tal
}
\local{Rio de Janeiro}
\data{Março de 2019}
\orientador{Beltrano Beltranoso}{Ph.D., do IME}
\coorientador{Tic Tac}%{M.Sc., do IME}
\instituicao{%
  Instituto Militar de Engenharia
  \par
  Seção de Computação
  \par
  Graduação em Engenharia de Computação
}
\tipotrabalho{Projeto Final de Curso}
% O preambulo deve conter o tipo do trabalho, o objetivo,
% o nome da instituição e a área de concentração 
\preambulo{Projeto Final de Curso de Graduação em Engenharia de Computação do
Instituto Militar de Engenharia.}
% ---


% ---
% Configurações de aparência do PDF final

% alterando o aspecto da cor azul
%\definecolor{blue}{RGB}{41,5,195}
\definecolor{blue}{RGB}{0,0,0}

% informações do PDF
\makeatletter
\hypersetup{
	%pagebackref=true,
	pdftitle={\@title},
	pdfauthor={\@author},
	pdfsubject={\imprimirpreambulo},
	pdfcreator={LaTeX with ime-abnTeX2},
	pdfkeywords={ime}{sua tag}{ip}{outra tag}{bla bla bla},
	colorlinks=true,		% false: boxed links; true: colored links
	linkcolor=blue,			% color of internal links
	citecolor=blue,			% color of links to bibliography
	filecolor=magenta,		% color of file links
	urlcolor=blue,
	bookmarksdepth=4
}
\makeatother
% ---

% ---
% Espaçamentos entre linhas e parágrafos 
% ---

% O tamanho do parágrafo é dado por:
%\setlength{\parindent}{1.3cm}

% Controle do espaçamento entre um parágrafo e outro:
%\setlength{\parskip}{0.2cm}  % tente também \onelineskip
\setlength{\parskip}{\onelineskip}

% ---
% compila o indice
% ---
\makeindex
% ---

% ----
% Início do documento
% ----
\begin{document}

% Retira espaço extra obsoleto entre as frases.
%\frenchspacing

% ----------------------------------------------------------
% ELEMENTOS PRÉ-TEXTUAIS
% ----------------------------------------------------------
% \pretextual

% ---
% Capa
% ---
\imprimircapa
%\input{pre_textuais/capa}
% ---

% ---
% Folha de rosto
% (o * indica que haverá a ficha bibliográfica)
% ---
\imprimirfolhaderosto*
%\input{pre_textuais/folha_de_rosto}
% ---

% ---
% Inserir a ficha bibliografica
% ---

% Isto é um exemplo de Ficha Catalográfica, ou ``Dados internacionais de
% catalogação-na-publicação''. Você pode utilizar este modelo como referência. 
% Porém, provavelmente a biblioteca da sua universidade lhe fornecerá um PDF
% com a ficha catalográfica definitiva após a defesa do trabalho. Quando estiver
% com o documento, salve-o como PDF no diretório do seu projeto e substitua todo
% o conteúdo de implementação deste arquivo pelo comando abaixo:
%
% \begin{fichacatalografica}
%     \includepdf{fig_ficha_catalografica.pdf}
% \end{fichacatalografica}
\imprimirfichacatalografica
{666.555}  % esse número deve ser montado com a ajuda da biblioteca do IME
{S455h}
{Silva, F., Tal, C.} % <último nome>, <abreviação do primeiro nome>
{Projeto de Fim de Curso (PFC)}
{%
  1. Curso de engenharia da computação - Projeto Final de Curso.
  2. Um Assunto.
  3. Outro Assunto.
  I. Silva, Cilano.
  II. Beltrano, Fulano.
  III. \@title.
  IV. Instituto Militar de Engenharia.}
{Técnicas de blablabla}
% ---

% ---
% Inserir errata
% ---
%\begin{errata}
%Elemento opcional da \citeonline[4.2.1.2]{NBR14724:2011}. Exemplo:
%
%\vspace{\onelineskip}
%
%FERRIGNO, C. R. A. \textbf{Tratamento de neoplasias ósseas apendiculares com
%reimplantação de enxerto ósseo autólogo autoclavado associado ao plasma
%rico em plaquetas}: estudo crítico na cirurgia de preservação de membro em
%cães. 2011. 128 f. Tese (Livre-Docência) - Faculdade de Medicina Veterinária e
%Zootecnia, Universidade de São Paulo, São Paulo, 2011.
%
%\begin{table}[htb]
%\center
%\footnotesize
%\begin{tabular}{|p{1.4cm}|p{1cm}|p{3cm}|p{3cm}|}
%  \hline
%   \textbf{Folha} & \textbf{Linha}  & \textbf{Onde se lê}  & \textbf{Leia-se}  \\
%    \hline
%    1 & 10 & auto-conclavo & autoconclavo\\
%   \hline
%\end{tabular}
%\end{table}
%
%\end{errata}
% ---

% ---
% Inserir folha de aprovação
% ---
%
% Isto é um exemplo de Folha de aprovação, elemento obrigatório da NBR
% 14724/2011 (seção 4.2.1.3). Você pode utilizar este modelo até a aprovação
% do trabalho. Após isso, substitua todo o conteúdo deste arquivo por uma
% imagem da página assinada pela banca com o comando abaixo:
%
% \includepdf{folhadeaprovacao_final.pdf}
%
% Por enquanto a classe precisa dos dois convidados
\convidadoum{Um Dois Três de Oliveira Quatro - D.Sc. do IME}{}
\convidadodois{Blau Bléu Bliu - Ph.D do IME}{}
\imprimirfolhadeaprovacao{19 de março de 2019}
% ---

% ---
% Dedicatória
% ---
%\begin{dedicatoria}
%   \vspace*{\fill}
%   \centering
%   \noindent
%   \textit{ Este trabalho é dedicado às crianças adultas que,\\
%   quando pequenas, sonharam em se tornar cientistas.} \vspace*{\fill}
%\end{dedicatoria}
% ---

% ---
% Agradecimentos
% ---
%\begin{agradecimentos}
%Os agradecimentos principais são direcionados à Gerald Weber, Miguel Frasson,
%Leslie H. Watter, Bruno Parente Lima, Flávio de Vasconcellos Corrêa, Otavio Real
%Salvador, Renato Machnievscz\footnote{Os nomes dos integrantes do primeiro
%projeto abn\TeX\ foram extraídos de
%\url{http://codigolivre.org.br/projects/abntex/}} e todos aqueles que
%contribuíram para que a produção de trabalhos acadêmicos conforme
%as normas ABNT com \LaTeX\ fosse possível.
%
%Agradecimentos especiais são direcionados ao Centro de Pesquisa em Arquitetura
%da Informação\footnote{\url{http://www.cpai.unb.br/}} da Universidade de
%Brasília (CPAI), ao grupo de usuários
%\emph{latex-br}\footnote{\url{http://groups.google.com/group/latex-br}} e aos
%novos voluntários do grupo
%\emph{\abnTeX}\footnote{\url{http://groups.google.com/group/abntex2} e
%\url{http://abntex2.googlecode.com/}}~que contribuíram e que ainda
%contribuirão para a evolução do \abnTeX.
%
%\end{agradecimentos}
% ---

% ---
% Epígrafe
% ---
%\begin{epigrafe}
%    \vspace*{\fill}
%	\begin{flushright}
%		\textit{``Não vos amoldeis às estruturas deste mundo, \\
%		mas transformai-vos pela renovação da mente, \\
%		a fim de distinguir qual é a vontade de Deus: \\
%		o que é bom, o que Lhe é agradável, o que é perfeito.\\
%		(Bíblia Sagrada, Romanos 12, 2)}
%	\end{flushright}
%\end{epigrafe}
% ---

% ---
% RESUMOS
% ---

% resumo em português
\setlength{\absparsep}{18pt} % ajusta o espaçamento dos parágrafos do resumo
\begin{resumo}

% seu texto
\lipsum[1]

% Segundo a \citeonline[3.1-3.2]{NBR6028:2003}, o resumo deve ressaltar o
% objetivo, o método, os resultados e as conclusões do documento. A ordem e a extensão
% destes itens dependem do tipo de resumo (informativo ou indicativo) e do
% tratamento que cada item recebe no documento original. O resumo deve ser
% precedido da referência do documento, com exceção do resumo inserido no
% próprio documento. (\ldots) As palavras-chave devem figurar logo abaixo do
% resumo, antecedidas da expressão Palavras-chave:, separadas entre si por
% ponto e finalizadas também por ponto.
%
% \textbf{Palavras-chaves}: latex. abntex. editoração de texto.
\end{resumo}

% resumo em inglês
\begin{resumo}[Abstract]
\begin{otherlanguage*}{english}

% seu texto
\lipsum[2]

%\vspace{\onelineskip}

%\noindent
%\textbf{Key-words}: latex. abntex. text editoration.
\end{otherlanguage*}
\end{resumo}

% resumo em francês 
%\begin{resumo}[Résumé]
% \begin{otherlanguage*}{french}
%    Il s'agit d'un résumé en français.
% 
%   \textbf{Mots-clés}: latex. abntex. publication de textes.
% \end{otherlanguage*}
%\end{resumo}

% resumo em espanhol
%\begin{resumo}[Resumen]
% \begin{otherlanguage*}{spanish}
%   Este es el resumen en español.
%  
%   \textbf{Palabras clave}: latex. abntex. publicación de textos.
% \end{otherlanguage*}
%\end{resumo}
% ---

% ---
% inserir o sumario
% ---
\pdfbookmark[0]{\contentsname}{toc}
\tableofcontents*
\cleardoublepage
% ---

% ---
% inserir lista de ilustrações
% ---
\pdfbookmark[0]{\listfigurename}{lof}
\listoffigures*
\cleardoublepage
% ---

% ---
% inserir lista de tabelas
% ---
%\pdfbookmark[0]{\listtablename}{lot}
%\listoftables*
%\cleardoublepage
% ---

% ---
% inserir lista de abreviaturas e siglas
% ---
%\begin{siglas}
%  \item[Fig.] Area of the $i^{th}$ component
%  \item[456] Isto é um número
%  \item[123] Isto é outro número
%  \item[lauro cesar] este é o meu nome
%\end{siglas}
% ---

% ---
% inserir lista de símbolos
% ---
%\begin{simbolos}
%  \item[$ \Gamma $] Letra grega Gama
%  \item[$ \Lambda $] Lambda
%  \item[$ \zeta $] Letra grega minúscula zeta
%  \item[$ \in $] Pertence
%\end{simbolos}
% ---



% ----------------------------------------------------------
% ELEMENTOS TEXTUAIS
% ----------------------------------------------------------
\textual

% ----------------------------------------------------------
% Introdução
% ----------------------------------------------------------

\chapter{Introdução}
\lipsum[3]

% TODO: capítulo para tema+problema

% Desenvolvimento
% ---------------

\chapter{Desenvolvimento}

\section{Foo}
\lipsum[4]

\section{Bar}
\lipsum[5]

% ----------------------------------------------------------
% PARTE - preparação da pesquisa
% ----------------------------------------------------------
%\part{Preparação da pesquisa}

% ----------------------------------------------------------
% Capitulo com exemplos de comandos inseridos de arquivo externo 
% ----------------------------------------------------------

%\include{abntex2-modelo-include-comandos}

% ----------------------------------------------------------
% Parte de revisão de literatura
% ----------------------------------------------------------
%\part{Referenciais teóricos}

% ---
% Capitulo de revisão de literatura
% ---
%\chapter{Lorem ipsum dolor sit amet}

% ---
%\section{Aliquam vestibulum fringilla lorem}
% ---

%\lipsum[1]

%\lipsum[2-3]

% ----------------------------------------------------------
% Resultados
% ----------------------------------------------------------
%\part{Resultados}

% ---
% primeiro capitulo de Resultados
% ---
%\chapter{Lectus lobortis condimentum}

% ---
%\section{Vestibulum ante ipsum primis in faucibus orci luctus et ultrices
%posuere cubilia Curae}
% ---

%\lipsum[21-22]

% ---
% segundo capitulo de Resultados
% ---
%\chapter{Nam sed tellus sit amet lectus urna ullamcorper tristique interdum
%elementum}
%
%\section{Pellentesque sit amet pede ac sem eleifend consectetuer}
%
%\lipsum[24]

% ---
% Finaliza a parte no bookmark do PDF, para que se inicie o bookmark na raiz
% ---
\bookmarksetup{startatroot}%
% ---

% ---
% Conclusão
% ---
%\chapter*[Conclusão]{Conclusão}
%\addcontentsline{toc}{chapter}{Conclusão}

%\lipsum[31-33]

% ----------------------------------------------------------
% ELEMENTOS PÓS-TEXTUAIS
% ----------------------------------------------------------
\postextual


% ----------------------------------------------------------
% Referências bibliográficas
% ----------------------------------------------------------
%\bibliographystyle{plainnat}%abbrvnat, unsrtnat, apsrev, rmpaps, IEEEtranN, achemso, rsc
%\bibliography{referencias}

% ----------------------------------------------------------
% Glossário
% ----------------------------------------------------------
%
% Consulte o manual da classe abntex2 para orientações sobre o glossário.
%
%\glossary

% ----------------------------------------------------------
% Apêndices
% ----------------------------------------------------------

% ---
% Inicia os apêndices
% ---
%\begin{apendicesenv}

% Imprime uma página indicando o início dos apêndices
%\partapendices

% ----------------------------------------------------------
%\chapter{Quisque libero justo}
% ----------------------------------------------------------

%\lipsum[50]

% ----------------------------------------------------------
%\chapter{Nullam elementum urna vel imperdiet sodales elit ipsum pharetra ligula
%ac pretium ante justo a nulla curabitur tristique arcu eu metus}
% ----------------------------------------------------------
%\lipsum[55-57]

%\end{apendicesenv}
% ---


% ----------------------------------------------------------
% Anexos
% ----------------------------------------------------------

% ---
% Inicia os anexos
% ---
%\begin{anexosenv}

% Imprime uma página indicando o início dos anexos
%\partanexos

% ---
%\chapter{Morbi ultrices rutrum lorem.}
% ---
%\lipsum[30]

% ---
%\chapter{Cras non urna sed feugiat cum sociis natoque penatibus et magnis dis
%parturient montes nascetur ridiculus mus}
% ---

%\lipsum[31]

% ---
%\chapter{Fusce facilisis lacinia dui}
% ---

%\lipsum[32]

%\end{anexosenv}

%---------------------------------------------------------------------
% INDICE REMISSIVO
%---------------------------------------------------------------------

\printindex

\end{document}
