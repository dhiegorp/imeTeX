\documentclass{imetex-inci}

%% Informações de Capa etc.
\author{Victor Villas Bôas Chaves}
\title{Algoritmos em Computação Quântica}
\date{\the\year}

% Comando que prepara o índice
\makeindex

\begin{document} 
\printFrontMatter

\chapter*{Resumo}
O presente trabalho apresenta um estudo sobre sistemas determinísticos, sistemas probabilísticos, sistemas quânticos, qubits e portas lógicas quânticas. Paralelamente ao estudo, foi implementado na linguagem de programação python programas para cálculos e simulações de sistemas quânticos.

\textbf{Palavras Chave:} Algoritmos, Computação Quântica

\chapter{Introdução}
\section{Justificativa}
Na computação clássica o computador é baseado na arquitetura de Von Neumann que faz uma distinção clara entre elementos de processamento e armazenamento de dados, isto é, possui processador e memória destacados por um barramento de comunicação, sendo seu processamento sequencial.
 
 Entretanto os computadores atuais possuem limitações, como por exemplo, na área de Criptografia e Segurança de Dados, onde não existem computadores com potência ou velocidade de processamento suficiente para suportar algoritmos de alta complexidade. Dessa forma surgiu a necessidade da criação de um computador diferente dos usuais que resolvesse problemas como a fatoração de números primos muito grandes, logaritmos discretos e simulação de problemas da Física Quântica.
 
 E assim os estudos em Computação Quântica se tornaram muito importantes e a necessidade do desenvolvimento de uma máquina extremamente eficiente se torna maior a cada dia.
 
 Na computação quântica a unidade de informação básica é o Bit quântico ou q-bit. O fato de a computação quântica ser tão poderosa está no fato de que além de assumir '0' ou '1' como na computação clássica, ela pode assumir ambos os estados '0' e '1' ao mesmo tempo.
 
 E é graças a essa propriedade da superposição de estados que motivou os estudos em computação quântica. Se na computação clássica o processamento é sequencial, na computação quântica o processamento é simultâneo.
 
 A computação quântica é um assunto que está começando a ser estudado na esfera global, embora o computador quântico seja incipiente, já existem alguns algoritmos quânticos, como por exemplo, o algoritmo de Shor para fatoração e algoritmo de Grover para acelerar o processo de procura em banco de dados.
 
 Os algoritmos quânticos abrem portas para complexidades inferiores aos já existentes, o que futuramente terá impacto direto em criptografia, assinatura digital, sistemas de cartões de crédito e segurança digital de modo geral, pois com tais algoritmos será possível tratar os problemas np-completos (para os quais não se conhece solução polinomial) por uma nova esfera.

\section{Objetivos}
Este projeto tem como objetivo a análise de caráter geral e o desenvolvimento de algoritmos computacionais quânticos, bem como a comparação com algoritmos da computação tradicional, para resolução de problemas da literatura convencional.

\chapter{Desenvolvimento}
\section{Cronograma}
O desenvolvimento da pesquisa tem o seguinte planejamento:
\begin{enumerate}
\item Pesquisa bibliográfica do princípio de funcionamento de um computador quântico.
\item Pesquisa sobre as diferenças na programação de algoritmos quânticos em relação à programação binária tradicional.
\item Análise dos algoritmos quânticos específicos.
\item Estudo comparativo dos algoritmos quânticos com os respectivos algoritmos do paradigma tradicional.
\end{enumerate}
Com base nessas atividades, o seguinte cronograma é previsto para o projeto:
\begin{table}[h]
\centering
\begin{tabular}{|c|c|c|c|c|}
\hline 
Atividade & 1º Trimestre & 2º Trimestre & 3º Trimestre & 4º Trimestre \\ 
\hline 
1 & X &   &   &   \\ 
\hline 
2 & X & X &   &   \\ 
\hline 
3 &   & X & X &   \\ 
\hline 
4 &   &   & X & X \\ 
\hline 
\end{tabular}
\caption{Distribuição de atividades}
\end{table}

\section{Atividades}
O livro utilizado para construir a base teórica necessária foi o Quantum Computing for Computer Scientists  \cite{yanofsky}, do qual os capítulos 1, 2, 3, 4 e 5 foram estudados. Três apresentações foram feitas para o orientador, das quais a primeira abrangeu os capítulos 1 e 2, a segunda abrangeu o capítulo 3 e a terceira o capítulo 4.

\chapter{Resultados}

\printBackMatter
\bibliography{bibliografia}
\end{document}